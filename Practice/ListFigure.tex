\documentclass[14pt, a4paper]{article} %14 pt indicates the font size of the prepared document
\usepackage[utf8]{inputenc} %indicates the encoding of the document
\usepackage{color} %this package enables the use of colors.
\usepackage{graphicx}

\title{Introductory Class on Latex}
\author{Mahjabin Nahar}
\date{\today}
\date{21 March 2021}

\begin{document}
\maketitle
\tableofcontents %this command creates the table of contents with all numbered sections, subsections, etc.
\pagebreak %This will force the rest of the document to start in another page.

\section{Section 1}
\label{sec:intro} %This will be used while referencing this section
This is the introduction section.

This is an itemized list.
\begin{itemize}
 \item item 1
 \begin{itemize}
     \item nested item 1
     \item nested item 2
 \end{itemize}
 \item item 2
 \item item 3
\end{itemize}

\subsection{Subsection 1.1}
\label{subsec:1.1} %This will be used while referencing this subsection
This is subsection 1.1.

This is an enumerated list.
\begin{enumerate}
 \item First item
 \item Second item
 \item This can also be nested.
 \begin{enumerate}
     \item nested item 1
     \item nested item 2
 \end{enumerate}
\end{enumerate}

\subsubsection{Subsubsection 1.1.1}
This is subsubsection 1.1.1.

This is a descriptive list.
\begin{description}
\item[Number 1] one
\item[Number 2] two
\end{description}

\pagebreak 

\subsection*{Unnumbered subsection}
This is an unnumbered subsection under Introduction. 
\begin{figure}[h]
	\centering
	\caption{This caption is at the top}
	\includegraphics{figure1.jpg}
	\label{fig:1}
	\caption{This is figure 1.}
\end{figure}


\section*{Unnumbered Section Example}
This section should not be numbered. Using * after the section specifier prevents this numbering. This will NOT show up in the Table of Contents. 

Referencing the figure used above: \ref{fig:1}.

\end{document}